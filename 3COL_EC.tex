\documentclass[12pt]{article}
\setlength\parindent{0em}
\setlength\parskip{0.75em}
\renewcommand\baselinestretch{1.3}
\usepackage[utf8]{inputenc}
\usepackage[legalpaper, margin=1in]{geometry}
\usepackage{amsmath}
\usepackage{amssymb}

\begin{document}
\textbf{Proof:} $3COL \le_p EC$.

Let $G = (V, E)$ be a graph. For a vertex $v \in V$, let $n(v)$ be the set of neighbors of $v$. For each vertex $v \in V$, for each color $c$, create a ``neighborhood'' set $\{u_{c} \mid u \in n(v)\} \cup \{v, v_c\}$. Additionally, for each vertex $v \in V$ and for each color $c$, create a ``slack'' set $\{v_c\}$. Then, let $X$ be the union of all of the sets constructed so far, and let $S$ be the set containing the sets constructed so far. Then, $G$ is 3-colorable iff an exact cover for $X$ can be constructed from sets in $S$.

Let $C$ be a 3-coloring of $G$. Then, a valid cover can be obtained by picking, for each $v \in V$, the neighborhood set that matches its color in $C$ (e.g., for a vertex $x$ colored $c$, the set containing $x$, $x_c$ and its neighbors $u_c$). Each $u_c$ cannot be covered more than once, since adjacent vertices must have different colors, therefore the corresponding neighborhood sets cannot intersect. The slack sets can be used to complete the cover.

Let $S^*$ be an exact cover. Then, a valid coloring is obtained by coloring each $v \in V$ with the color that matches the neighborhood set containing $v$ in $S^*$. For each $v \in V$, there is exactly one neighborhood set in $S^*$. Indeed, the only way to obtain an element $v$ is to select one of its neighborhood sets, and at most one can be selected since they all contain $v$. Furthermore, neighborhood sets in $S^*$ for adjacent vertices cannot intersect, so they must have different colors. The slack sets in $S^*$ can simply be ignored.
\end{document}


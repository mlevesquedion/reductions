\documentclass{amsart}
\usepackage[utf8]{inputenc}

\begin{document}
\noindent
\textbf{Proof:} $3SAT \le_T CLIQUE$\\

Let $\Phi$ be a $3SAT$ formula. Given an algorithm that solves $CLIQUE$, we can also solve $3SAT$ by creating a graph $G$ wherein each literal in a clause is represented by a vertex connected to every non-conflicting literal from other clauses, and asking for a clique of size $k = \text{the number of clauses in } \Phi$. The only conflicting literals are $x_i$ and its negation, $\lnot x_i$, for all $i$.

If there is a satisfying assignment, then let $T$ be the set of literals that evaluate to $True$. There is at least one such literal in each clause, therefore $|T| \ge k$. Furthermore, since each of these $T$ literals is allowed to be $True$ simultaneously, then they are not conflicting and therefore are all connected by edges in $G$, forming a clique of size $k$.

If there is a clique $C$ in $G$, then $C$ must contain exactly one vertex for each clause in $\Phi$, because the vertices representing the literals from each clause are not connected to each other. Furthermore, since the vertices in $G$ represent non-conflicting literals, a satisfying assignment is obtained by setting the underlying variables such that each of them is $True$.

\hfill$\qed$

\end{document}

